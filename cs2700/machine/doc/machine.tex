\documentclass{article}
\author {M.~W.~Schulte}
\title {Introduction to Schulte's Machine}
\def\instr#1#2#3#4#5#6{\hspace*{2em}\makebox[1.5in][l]{\bf #1} {\tt #2}\\
						\hspace*{5em}\makebox[1in][l]{args= #3}
						\hspace*{.75in}\makebox[2in][l]{stack= #4 -- #5}
						\begin{quotation}#6\end{quotation}}
\def\bs{$\backslash$}
\begin{document} \parindent 0pt \parskip 1ex
\include {mydefs}\fullpage
\section{The Machine Architecture}
The machine, to the programmer, looks as if it were composed of
three sections:\begin {itemize}
\item Code Space:
a section of byte-addressable RAM, addressed by 64-bit addresses.
All executable code must be in this section; the operating
system can mark this section read-only in some cases.
There is an internal register PC (not accessible to the programmer)
which points to the next instruction to execute.
Any data in the code section is accessed
either by explicit 64-bit address, or by 16-bit offset from PC.
In all cases, this offset is from the PC of the byte following the word
containing the offset [PC of the next instruction].
\item Data Space:
a section of byte-addressable RAM, addressed by 64-bit addresses.
There is provision for "short" addresses;  short addresses are 16-bit
addresses, and behave as a 64-bit address with the top 48 bits set to zero.
This section is meant for use by the static variables and the heap.
All objects are access by explicit address.
\item Stack: a section of word-addressable stack.
This serves as both the work space for all operations, and the storage
for the local variables in all modules.
There is an internal register SP (not accessible to the programmer)
to access the top of the stack; there are instructions to
push items of the stack and to change the value of SP.
There is also an internal array of frame pointers (also not directly
accessible to the programmer) with instructions to set and release frames
by manipulating the values in the frame pointer array.
\end{itemize}

\section{Data Formats}
For all sections of memory, the data is stored in a "little-endian" manner:
the least significant byte (or word) is stored at the smallest address;
the successive bytes (or words) are stored at progressively higher
addresses.  Since the code and data spaces are byte-addressable, but the
stack is word-addressible, we will not define the contents of the upper
byte on the stack of the result of the movement of a byte onto the stack
(and the programmer should not rely on any particular value),
and the upper byte is ignored on a movement of a byte off of the stack.

There are three integer formats: character, integer, and long integer.
A character integer is a byte long, and is meant to allow operations
on characters, \`{a} la C characters.  It also allows character comparisons
in any language.
A regular integer is a word long.  It is a twos-complement signed integer.
A long integer is a quadword (64 bits) long.  It also is a twos-complement
signed integer.

There is one floating point format.  It occupies a longword, and uses
the floating point format on the underlying machine (yes, I know this is
not portable.)

\section{The Instructions}
In many of the machine instruction formats, there is referrence to
the quantity {\bf l} with a corresponding field {\tt ll} in the
machine format; this is a field to indicate the length of the
item:
\begin{tabular}[t]{l|l}{\bf l}&{\tt ll}\\W&00\\L&01\\Q&10\\B&11\\O&11
\end{tabular}; O is used for {\bf DUPl} and {\bf SWAPl}, B is used with
the other instructions.\\
The instruction field {\bf t} is also used, with the machine format
field {\tt tt}; this indicates data type:
\begin{tabular}[t]{l|l}{\bf t}&{\tt tt}\\I&00\\L&01\\F&10\\C&11
\end{tabular}

\small
\subsection{Reserved Instructions}
\instr{HALT}		{0000 0000} {} {} {}
	{Halts the machine.  For non-priveledged processes, the exception
	handler terminates the program.}
\instr{SYSCALL}	{0000 0001} {} {call dependent values \bs\ type}
		{call dependent results}
	{\mbox{}}
\instr{SYSRET}	{0000 0010} {} {} {}
	{For internal use only}
%\instr{TRACE}	{0000 0011} {} {} {}
%	{For internal use only}

\subsection{PC Relative Instructions}
\instr{RGETl}	{0001 00ll}	{w\_PC\_offset} {} {value}
	{Get the value of a variable and put it on the top of the stack.}
\instr{RPUTl}	{0001 01ll}	{w\_PC\_offset} {value} {}
	{Put the value on the top of the stack into a variable.}
\instr{RGOTO}	{0001 1000} {w\_PC\_offset} {} {}
	{Go to the instruction whose code space address is the sum of the
	w\_PC\_offset and the PC of the next instruction.}
\instr{RGOSUB}	{0001 1001} {w\_PC\_offset} {} {q\_ret\_addr}
	{Push the address of the next instruction on the stack, then go to 
	the instruction whose address is the sum of that address and 
	w\_PC\_offset.}
\instr{RGOZ}	{0001 1010} {w\_PC\_offset} {w\_flag} {}
	{If w\_flag is a zero word [all zeroes], then go to the instruction 
	whose address is the sum of the w\_PC\_offset and the address of the 
	next instruction.}
\instr{RPUSHA}	{0001 1011} {w\_PC\_offset} {} {q\_c\_addr}
	{Push the code space address formed by adding the address of the next 
	instruction and the w\_PC\_offset.}
\instr{PUSHl}		{0001 11ll}	{value} {} {value}
	{Push the argument (whose length is $l$) onto the top of the stack.  
	[Push Immediate]}

\subsection{Code Space Instructions}
\instr{CGETl}	{0010 00ll}	{} {q\_c\_address} {value}
	{Get the value at the location whose code space address is on the 
	top of the stack, and put it on the top of the stack.}
\instr{CPUTl}	{0010 01ll}	{} {q\_c\_address \bs\ value} {}
	{Put the value on the top of the stack into the location
	whose code space address is the second on the stack.}
\instr{GOTO}	{0010 1000} {} {q\_c\_addr} {}
	{Go to the instruction whose code space address is on the 
	top of the stack.}
\instr{RET}		{0100 1000} {} {q\_c\_addr} {}
	{Same as GOTO; introduced for conveience of the programmer.}
\instr{GOSUB}	{0010 1001} {} {q\_c\_addr} {q\_ret\_addr}
	{Pop a quad-word from the stack, push the code space address of the 
	next instruction, then go to the instruction whose code space address
	is the quad-word that was originally popped.}
\instr{GOZ}		{0100 1010} {} {q\_c\_addr \bs\ w\_flag} {}
	{If w\_flag is a zero word [all zeroes], then goto the instruction
	whose code space address is the second on the stack.}

\subsection{Data Space Instructions}
\instr{GETSl}	{0011 00ll}	{} {w\_d\_address} {value}
	{Get the value at the location whose (16-bit) data space address
	is on the top of the stack, and put it on the top of the stack.}
\instr{PUTSl}	{0011 01ll}	{} {w\_d\_address \bs\ value} {} 
	{Put the value on the top of the stack into the location whose
	(16-bit) data space location is the second on the stack.}
\instr{GETFl}	{0011 10ll}	{} {q\_d\_address} {value}
	{Get the value at the location whose (64-bit) data space address 
	is on the top of the stack, and put it on the top of the stack.}
\instr{PUTFl}	{0011 11ll}	{} {q\_d\_address \bs\ value} {} 
	{Put the value on the top of the stack into the location whose 
	(64-bit) data space location is the second on the stack.}

\subsection{Frame Instructions}
\instr{LGETSl}  {0100 00ll} {} {w\_frame \bs\ w\_offset} {value}
	{Get the value at the location whose frame is second on stack and
	whose (16-bit) signed offset is on the top of the stack, and put
	the value on the top of stack.}
\instr{LPUTSl}  {0100 01ll} {} {w\_frame \bs\ w\_offset \bs\ value} {}
	{Put the value on the top of the stack into the location whose
	frame is third on stack and whose (16-bit) offset is second on
	the stack.}
\instr{LGETFl}  {0100 10ll} {} {w\_frame \bs\ q\_offset} {value}
	{Get the value at the location whose frame is second on stack and
	whose (64-bit) signed offset is on the top of the stack, and put
	the value on the top of stack.}
\instr{LPUTFl}  {0100 11ll} {} {w\_frame \bs\ q\_offset \bs\ value} {}
	{Put the value on the top of the stack into the location whose
	frame is third on stack and whose (64-bit) offset is second on
	the stack.}

\subsection{Arithmetic Instructions}
\instr{ADDt}    {0101 00tt} {} {value1 \bs\ value2} {result}
	{Take top two values off stack, form the sum value1 + value2,
	and put result onto stack.}
\instr{SUBt}    {0101 01tt} {} {value1 \bs\ value2} {result}
	{Take top two values off stack, form the difference value1 - value2,
	and put result onto stack.}
\instr{MULt}    {0101 10tt} {} {value1 \bs\ value2} {result}
	{Take top two values off stack, form the product value1 * value2,
	and put result onto stack.}
\instr{DIVt}    {0101 11tt} {} {value1 \bs\ value2} {result}
	{Take top two values off stack, form the quotient value1 / value2,
	and put result onto stack.  For types I, L, and C, the result is
	the quotient with no remainder; for the type F, the result is the 
	representation of the real-number division.  (Division of long integers
	is not currently supported.)}

\subsection{Bit Instructions}
\instr{ANDl}   {0110 00ll} {} {value1 \bs\ value2} {result}
	{Take top two values off stack, form value1 AND value2 bit by bit,
	and put result onto stack.}
\instr{ORl}   {0110 01ll} {} {value1 \bs\ value2} {result}
	{Take top two values off stack, form value1 OR value2 bit by bit,
	and put result onto stack.}
\instr{XORl}   {0110 10ll} {} {value1 \bs\ value2} {result}
	{Take top two values off stack, form value1 XOR value2 bit by bit,
	and put result onto stack.}
\instr{NOTl}   {0110 11ll} {} {value} {result}
	{Take the top value from the stack, complement all the bits, and
	put result onto stack.}

\subsection {Stack Manipulation}
\instr{CSFTl}   {0111 00ll} {} {value \bs\ amt} {result}
	{Do a "circular shift".  If the value of amt is positive,
	shift to the left; if the value of amt is negative, shift to the right.
	every bit shifted off one end is replaced at the other end.}
\instr{ASFTl}   {0111 01ll} {} {value \bs\ amt} {result}
	{Do an "arithmetic shift".  If the value of amt is positive, shift 
	the value to the left, dropping the left-most bits from the value
	and placing 0's on the right side.  If the value of amt is negative,
	shift the value to the right with sign extention.}
\instr{TSTEQt}   {0111 10tt} {} {value} {flag}
	{If the value on the stack is equal to zero, push TRUE (-1), 
	else push FALSE (0).}
\instr{TSTLTt}   {0111 11tt} {} {value} {flag}
	{If the value on the stack is less than zero, push TRUE (-1), 
	else push FALSE (0).}
\instr{DUPl}   {1000 00ll} {} {value} {value \bs\ value}
	{Duplicate the value on the top of the stack.  This instruction uses
	the length specifier {\bf O} in place of {\bf B}; this is to allow
	the programmer to duplicate a pair of quadwords.}
\instr{SWAPl}   {1000 01ll} {} {value1 \bs\ value2} {value2 \bs\ value1}
	{Swap the top two elements oh the stack.  This instruction also uses
	the length specifier {\bf O} in place of {\bf B}; this allows the
	programmer to swap a pair of quadwords.}
\instr{CSHPx}   {1000 100y} {} {amt} {}
	{Add the signed value on the top of the stack to the Stack Pointer.
	The value of {\bf x} can be {\bf S} ({\tt y} = 0) to indicated that
	the value is a word value (sign extended to 64 bits) or 
	{\bf x} can be {\bf F} ({\tt y} = 1) to indicate that 
	the value is a quadword.}
\instr{NEWFR}   {1000 1010} {frame} {} {ofp}
	{Take the next word in the instruction stream as an index into the 
	Frame Pointer array.  Push the value at that index in the Frame Pointer 
	array on the stack, then
	insert the (new) value of SP into that location in the FP array.}
\instr{RESFR}   {1000 1011} {frame} {ofp \bs\ local stuff} {}
	{Take the next word in the instruction stream as an index into the
	Frame Pointer array.  Replace SP by the value at that index in the FP
	array (effectively discarding the "local stuff"), then pop the (new) 
	top of the stack into that location in the FP array.}

\subsection {Conversion Instructions}
\instr{CVTt$\mbox{}_1$t$\mbox{}_2$}
			{1001 t$\mbox{}_1$t$\mbox{}_1$t$\mbox{}_2$t$\mbox{}_2$}
			{} {value} {value}
	{Convert the value of type $t_1$ to the equivalent value of type $t_2$.
	If $t_1$ and $t_2$ are both integer types, then if $t_1$ is a smaller
	type than $t_2$, the conversion is by sign extention, else the conversion
	is by truncation.  If $t_1$ is an integer type and $t_2$ is {\tt F}, then
	then corresponding floating value is produced, with possible loss of the
	less significant digits (every integer type will fit into a float).
	If $t_1$ is {\tt F} and $t_2$ is an integer type, then conceptually the
	float is converted to an infinite precision integer, then truncated if
	need be to fit into type $t_2$. ($t_1$ and $t_2$ are NOT the same type.)}
\instr{ADRD}		{1001 0000} {} {q\_d\_address} {q\_universal\_address}
	{The quad-size data address is converted into a quad-size generic address
	(actually, the same address).}
\instr{ZEXWQ}		{1001 0101} {} {w\_value} {q\_value}
	{The word-size value on the stack is extended with zeros to a quad-size
	value.  (The initial and final values are identical as unsigned integers.)}
\instr{ADRC}		{1001 1010} {} {q\_c\_address} {q\_universal\_address}
	{The quad-size code address is converted into a quad-size generic address.}
\instr{ADRF}		{1001 1111} {} {q\_d\_address} {q\_universal\_address}
	{The frame address with quad-size offset is converted into a quad-size
	generic address.  (Note: the frame number of any frame address is always
	a word.)}

\end{document}
