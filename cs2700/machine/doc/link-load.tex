\documentclass {article}
\author {M.~W.~Schulte}
\title {The Format of the {\tt .o} File}
\include {mydefs}\fullpage
\begin{document}
\maketitle

The object file is an ASCII file containing  the generated code with 
intermixed directives to the linker/loader.

The instructions can be either written as approved memonics (in all upper
case) or as the equivalent numbers.  The arguements are numbers or symbols.
the forward references are handled automatically.

A number followed immediately by a colon (``:'') 
(no space between number and colon) is taken as a command 
to change the next storage location (PC)  to the number 
plus the start address of the module [specify the load address].  
The initial load address is taken to be location 0 in the code section. 

Ther are a series of linker/loader directives:\begin {description}
\item [{\tt .CODE}] indicates that the loading is to be done to the 
code section [the initial default].  
\item [{\tt .DATA}] indicates that 
the loading is to be done to the data section.
\item [{\tt .ENTRY}] followed by a number or symbol indicates the 
(relocatable) entry point.
\item [{\tt .EXTERN}] followed by a name and a parenthesized
list of numbers indicates that the name is an external symbol used at the
relative code locations in the list of value.
\item [{\tt .RELOCs}] where {\tt s} can be {\tt B}, {\tt W}, {\tt L}, or 
{\tt Q}, followed by a parenthesized list of numbers indicates
that the (relocatable) locations specified in the list 
in the current section need to be relocated.
\item [{\it name}:] (no space between name and colon) denotes a global symbol 
whose value is the (relocatable) current location in the current section.
The loader complains if you try to redefine the name, and uses the first 
occurence.
\item [{\it name}=] (no space before the equal sign)
followed by a name or a number 
denotes a global symbol whose value is value of arguement.
You can redefine the name.
\end {description}
\pagebreak

\zB,\\
SIZE = 512 .CODE 0: LPUTW HALT\\ 
A: PUSHW 0 PUSHW B PUSHW 0 PUSHW Z LGETW PUSHW SIZE MULTI \\
PUSHQ 0 GOTO\\
.ENTRY A .DATA 4: A: 32W 16: B: 25678912L

sets SIZE to the value 512, starts the CODE section (done by default also),
puts a series of instructions into the code section, declares the symbol A 
to have value 2 [PC at that point is 2], declares that the entry point is 
the label A [2 in the CODE section], switches to the DATA section, moves to 
location 4, declares a label Z with that value and puts the word 32 there, 
then moves to location 16, declares a label B and 
puts the longword 25678912 there.
\end{document}
