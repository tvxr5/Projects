\documentclass {report}
\author {M.~W.~Schulte}
\include {mydefs}\fullpage
\begin {document} \parindent 0em \parskip 1em
\section {The Commands Available}

The program presents a series of menus to guide you through
a session. 

On the main menu, you see:\begin {description}
\item[Q] to Quit the emulator
\item[L list] to Load modules.  You follow the letter with a list of files
that contain the modules desired, in the LINK-LOAD format.  Further L
commands append their modules to the modules already loaded.
\item[ST \#] to single STep the program.  You follow the command with the 
number of instructions to step.  
\item[R] to Run the program to the end.  Typing a control-C interrupts 
the run.
\item[ZP] to erase the modules already entered. (Zero Program).
\item[SH] to display the SHow menu.
\item[CH] to display the CHange menu.
\item[BK] to display the BreaKpoint menu.
\end {description}

On the show menu, you see:\begin {description}
\item[SI] Show next Instruction to execute
\item[SC \# \#] Show contents of Code space.  
The next word is the start location, either as a number or the name of 
a symbol in the symbol table.  The final word is the ending value, 
either as a number, the name of a symbol in the symbol table, or a 
plus sign and the number of entries to display 
[no space between the plus sign and the number].
\item[SD \# \#] Show contents of Data space.  
The next word is the start location, either as a number or the name of 
a symbol in the symbol table.  The final word is the ending value, 
either as a number, the name of a symbol in the symbol table, or a 
plus sign and the number of entries to display 
[no space between the plus sign and the number].
\item[SF \# \#] Show a Frame variable.  You follow the command with
the frame number and the offset.  It show only the word at that address.
\item[SS] Show the Stack (the top 20 elements if the stack contains more).  
The entries are words; TOS is to the right.
\item[SY] show the SYmbol table (linker's symbol table).
\end {description}

On the change menu, you see:\begin {description}
\item[CI] Change the next Instruction. Not Implemented.
\item[CC \#] Change the Code space. 
You follow with either a code space
address or the name of a symbol in the symbol table.  The emulator will 
respond with the address and current contents in hex of that address; 
you type the new contents in hex (you must type something, 
even if it is the same).
It repeats this with successive locations, until you type the escape 
key as the new value.  (That location doesn't change.)
\item[CD \#] Change the Data space. Not Implemented.
You follow with either a data space
address or the name of a symbol in the symbol table.  The emulator will 
respond with the address and current contents in hex of that address; 
you type the new contents in hex (you must type something, 
even if it is the same).
It repeats this with successive locations, until you type the escape 
key as the new value.  (That location doesn't change.)
\item[CF] Change a Frame variable. Not Implemented.
\item[CS] Change the Stack.  Not Implemented.
\end {description}

On the breakpoint menu, you see:\begin {description}
\item[BT \#] Breakpoint seT.  You follow with either the code space 
address or the name of a symbol in the symbol table of the location 
at which you wish a breakpoint.
\item[BS] Breakpoint Show.  Prints all the active breakpoints.
\item[BC \#] Breakpoint Change.  You follow with the address or symbol
name of the breakpoint you want removed.
\item[BZ] Breakpoint Zero.  Delete all breakpoints
\end {description}

The following are available, but are not on any menu.\begin {description}
\item[M] display the main Menu.
\item[SL] to enter SiLent mode. Menus are not displayed, nor are the
instructions as they are executed.
\item[V] to enter Verbose mode.  Menus are displayed, as are instructions 
as they execute.
\end {description}



\section {The Format of the LINK-LOAD File}

The object file is an ASCII file containing  the generated code with 
intermixed directives to the linker/loader.

The instructions can be either written as approved memonics (in all upper
case) or as the equivalent numbers.  The arguments are numbers or symbols.
the forward references are handled automatically.

A number followed immediately by a colon (``:'') 
(no space between number and colon) is taken as a command 
to change the next storage location (PC)  to the number 
plus the start address of the module [specify the load address].  
The initial load address is taken to be location 0 in the code section. 

There are a series of linker/loader directives:\begin {description}
\item [{\tt .CODE}] indicates that the loading is to be done to the 
code section [the initial default].  
\item [{\tt .DATA}] indicates that 
the loading is to be done to the data section.
\item [{\tt .ENTRY}] followed by a number or symbol indicates the 
(relocatable) entry point.
\item [{\tt .EXTERN}] followed by a name and a parenthesized
list of numbers indicates that the name is an external symbol used at the
relative code locations in the list of value.
\item [{\tt .RELOCs}] where {\tt s} can be {\tt B}, {\tt W}, {\tt L}, or 
{\tt Q}, followed by a parenthesized list of numbers indicates
that the (relocatable) locations specified in the list 
in the current section need to be relocated.
\item [{\it name}:] (no space between name and colon) denotes a global symbol 
whose value is the (relocatable) current location in the current section.
The loader complains if you try to redefine the name, and uses the first 
occurrence.
\item [{\it name}=] (no space before the equal sign)
followed by a name or a number 
denotes a global symbol whose value is value of argument.
You can redefine the name.
\end {description}
% \pagebreak

\zB,\\
SIZE= 512 .CODE 0: LPUTW HALT\\ 
A: PUSHW 0 PUSHW B PUSHW 0 PUSHW Z LGETW PUSHW SIZE MULTI \\
PUSHQ 0 GOTO\\
.ENTRY A .DATA 4: a: 32W 16: b: 25678912L

sets SIZE to the value 512, starts the CODE section (done by default also),
puts a series of instructions into the code section, declares the symbol A 
to have value 2 [PC at that point is 2], declares that the entry point is 
the label A [2 in the CODE section], switches to the DATA section, moves to 
location 4, declares a label Z with that value and puts the word 32 there, 
then moves to location 16, declares a label B and 
puts the longword 25678912 there.
\end{document}
